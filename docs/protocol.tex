\documentclass[12pt, a4paper]{article}
\usepackage[utf8]{inputenc}
\usepackage{parskip}

\title{Bank protocol}
\author{Wouter Bruggeman, Rick van Vonderen}
\date{\today}
\renewcommand{\contentsname}{Table of contents}

\begin{document}
\maketitle
\newpage

\tableofcontents
\newpage

\section{Introduction}
None!
\newpage

\section{Protcol}
\subsection{Block scheme}
Out protocol contains a few blocks. The blocks are defined below:\\
\\
For the requests we're using this scheme:\\
0: Type\\
1: Size\\
2: Data\\

For the response from the server we're usign this scheme:\\
0: Type\\
1: Size\\
2: Status\\
3: Data\\

Size defines the size of the data block in bytes.

\newpage

\subsection{Explanation of type}
The type block has a size of 1 byte. This byte is defined in the protocol header as
an enum.\\\\
This enum has the following options:\\
0: ping\\
\textbf{1: ping\_r}\\
2: login\_request\\
\textbf{3: login\_r}\\
4: logout\\
\textbf{5: logout\_r}\\
6: account\_list\\
\textbf{7: account\_list\_r}\\
8: user\_data\\
\textbf{9: user\_data\_r}\\
10: pageview\\
\textbf{11: pageview\_r}\\
12: transaction\\
\textbf{transaction\_r}\\

All items in \textbf{bold} are responses. The other items are requests from the client.

\subsection{Explanation of status}
Status can be:\\
\begin{itemize}
	\item TIMEOUT (-2)
	\item INVALID (-1)
	\item FAIL (0)
	\item OK (1)
\end{itemize}

\subsection{Explanation of data}
The format of the data blocks is defined by the packet type.\\

\subsubsection{0: ping}
Data[0] = Random byte.\\

Ping request. First byte in data block has to be returned from the server after adding 1.
Size must be 1.

\subsubsection{1: ping\_r}
Status = OK;
Data[0] = Data[0] from ping request + 1.

\subsubsection{2: login\_request}
Data[0-11] = Card UID.\\
Data[12-?24?] = Pincode in raw byte format(Not in char +48 display format)\\
Pincode length has to be between 4 and 12.

\subsubsection{3: login\_r}
Status = TIMEOUT/INVALID/OK/FAIL
Data[0] = Byte containing the failed attempts when status == FAIL.

\subsubsection{4: logout}
This request has no data.

\subsubsection{5: logout\_r}
This request has no reply data.

\subsubsection{6: account\_list}
This request has no data.

\subsubsection{7: account\_list\_r}
Data contains blocks for iban numbers including the balance and type.
Block is 43 in length:\\
Block[0-33] = IBAN\\
Block[34-41] = Balance\\
Block[42] = Type\\

\subsubsection{8: user\_data}
This request has no data.

\subsubsection{9: user\_data\_r}
Data[0-99] = Fistname\\
Data[100-199] = Lastname\\
Data[200-204] = Year\\
Data[205] = Month\\
Data[206] = Day\\

\subsubsection{10: pageview}
Data[0] = PAGE(int)

\subsubsection{11: pageview\_r}
This response has no data.

\subsubsection{12: transaction}
Data[0-33] = From Iban\\
Data[34-67] = To Iban\\
Data[68-75] = Amount\\

\subsubsection{13: transaction\_r}
Status = TIMEOUT/INVALID/OK/FAIL
Data[0] = Fail reason

Fail reason can be:
\begin{itemize}
	\item 0: Not enough balance
	\item 1: To Iban invalid
	\item 2: From Iban invalid
	\item 3: Both Ibans invalid
\end{itemize}

\end{document}
