\documentclass[12pt, a4paper]{article}
\usepackage[utf8]{inputenc}
\usepackage{parskip}

\title{Bank protocol}
\author{Wouter Bruggeman, Rick van Vonderen}
\date{\today}
\renewcommand{\contentsname}{Table of contents}

\begin{document}
\maketitle
\newpage

\tableofcontents
\newpage

\section{Introduction}
For this server we've made a TCP protocol.This protocol defines all requests
and responses for the server and client.\\
Currently, there is no support for server to server communication. We'll add that later.\\
\newpage

\section{Protocol}
\subsection{Block scheme}
The protocol contains a few blocks of data. The blocks are defined below:\\
\\
For the requests we're using this scheme:\\
\begin{tabular}{| p{2cm} | p{5cm} |}
	\hline
	\textbf{Block nr.} & \textbf{Block name} \\ \hline
	0 & Type \\ \hline
	1 & Size \\ \hline
	2 & Data \\ \hline
\end{tabular}

For the response from the server we're using this scheme:\\
\begin{tabular}{| p{2cm} | p{5cm} |}
	\hline
	\textbf{Block nr.} & \textbf{Block name} \\ \hline
	0 & Type \\ \hline
	1 & Size \\ \hline
	2 & Status \\ \hline
	3 & Data \\ \hline
\end{tabular}

\newpage

\subsection{Explanation of type}
The type block has a size of 1 byte. This byte defines the type of the data block.\\
Type can contain one of the following:\\

\begin{enumerate}
	\setcounter{enumi}{-1}
	\item ping
	\item \textbf{ping\_r}
	\item login\_request
	\item \textbf{login\_r}
	\item logout
	\item \textbf{logout\_r}
	\item account\_list
	\item \textbf{account\_list\_r}
	\item user\_data
	\item \textbf{user\_data\_r}
	\item pageview
	\item \textbf{pageview\_r}
	\item transaction
	\item \textbf{transaction\_r}
\end{enumerate}

All items in \textbf{bold} and suffixed with \_r are responses from the server.
The other items are requests from the client.

\subsection{Explanation of size}
The size byte contains the amount of bytes in the data block.\\
The maximum number stored in a byte is 255. This way the data block
cannot be larger than 255 bytes.\\ The total size of a request
packet cannot be larger than 255 + 2 = 257 bytes.

\subsection{Explanation of status}
This byte will only be used in response packet wich are only send by the server.\\
Status can contain one of the following:\\
\begin{itemize}
	\item TIMEOUT (-2)
	\item INVALID (-1)
	\item FAIL (0)
	\item OK (1)
\end{itemize}

\subsection{Explanation of data}
The format of the data block is defined by the packet type, wich we'll describe below.

\subsubsection{0: ping}
Data[0] = Random byte.\\

This is just a ping request. One of the many possible usecases could be latency testing.

\subsubsection{1: ping\_r}
Status = OK;
Data[0] = Data[0] from ping request + 1.

\subsubsection{2: login\_request}
Data[0-11] = Card UID.\\
Data[12-?24?] = Pincode in raw byte format(Not in char +48 display format)\\
Pincode length has to be between 4 and 12.

The card UID must be in hexadecimal format.\\

\subsubsection{3: login\_r}
Status = TIMEOUT/INVALID/OK/FAIL\\
Data[0] = Byte containing the amount of failed attempts.\\
\\
This response contains information about the login attempt.\\
Every time the user fails to login, a counter will be incremented.
After 3 failed attempts the card is blocked.

\subsubsection{4: logout}
Ask the server if we can logout.\\
This request has no data.

\subsubsection{5: logout\_r}
Respond to the logout request. Usually the status byte contains 'OK'.\\
This request has no reply data.

\subsubsection{6: account\_list}
This request has no data.

\subsubsection{7: account\_list\_r}
Data contains account\_blocks for iban numbers including the balance and type.\\
The size of an account\_block is 43 bytes.
Every account\_block is added after the other in the data block.\\

To calculate the amount of accounts the user has, use the following formula:\\
amount = size / 43;\\
Size is the size of the packet.\\
\\
Account\_block[0-33] = Iban\\
Account\_block[34-41] = Balance\\
Account\_block[42] = Type\\

\subsubsection{8: user\_data}
Request information about the currently logged in user.\\
This request has no data.

\subsubsection{9: user\_data\_r}
This response contains information about the currently logged in user.\\
Firstname and Lastname are char arrays, Year is an integer of 8 bytes.\\
Day and month are unsigned bytes.\\
\\
Data[0-99] = Fistname\\
Data[100-199] = Lastname\\
Data[200-204] = Year\\
Data[205] = Month\\
Data[206] = Day\\

\subsubsection{10: pageview}
Data[0] = Page\\
\\
Page contains the index(byte) of the current page.\\
More information about pages will be added soon.

\subsubsection{11: pageview\_r}
This response has no data.

\subsubsection{12: transaction}
Request a transaction.\\
Amount has a size of 8 bytes and is unsigned.

Data[0-33] = From Iban\\
Data[34-67] = To Iban\\
Data[68-75] = Amount\\

\subsubsection{13: transaction\_r}
This response contains information about the requested transaction.\\
\\
Status = TIMEOUT/INVALID/OK/FAIL\\
Data[0] = Fail reason\\
\\
Fail reason can be:
\begin{itemize}
	\item 0: Not enough balance
	\item 1: To Iban invalid
	\item 2: From Iban invalid
	\item 3: Both Ibans invalid
\end{itemize}

\end{document}
